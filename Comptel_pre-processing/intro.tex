Introduction
These proceedings contain the papers presented at the 8th Workshop on the Use of Computational Methods in the Study of Endangered Languages (ComputEL-8), held on March 4–5, 2025 in Honolulu, Hawai‘i. The workshop is co-located with the 9th International Conference on Language Documentation & Conservation (ICLDC9) and offers hybrid attendance options, enabling participants to join either in-person or remotely.

As the name implies, this is the eighth workshop dedicated to the intersection of computational tools and endangered language research. The inaugural event took place at the Association for Computational Linguistics (ACL) main conference in Baltimore, Maryland in 2014. Subsequent workshops have been co-located with the International Conference on Language Documentation & Conservation at the University of Hawai‘i at Mānoa (2017, 2019, 2021, 2023) or ACL-related venues (2022 in Dublin, Ireland; 2024 in St. Julians, Malta). We are delighted to continue this tradition by returning to Honolulu, marking the fifth time the workshop has been held alongside ICLDC.

The primary aim of ComputEL-8 is to bring together computational researchers, documentary linguists, and community language practitioners. By uniting these diverse groups, the workshop fosters a collaborative environment for exchanging ideas, methods, and resources that support the documentation and revitalization of endangered languages. The organizers are gratified by the variety of contributions, reflecting the importance of collaborative efforts across different disciplines and communities.

This year, we received 45 submissions in the form of papers or extended abstracts. Following a thorough review process, 30 were accepted. In addition, 3 presentations formed our special session, titled “Building Tools Together,” which focused on strategies for joint development of language resources and technologies.

We extend our appreciation to all authors for their submissions and to the Program Committee for the thoughtful review of each proposal. We also thank the ICLDC9 organizers for their assistance in hosting this workshop. We hope that ComputEL-8 sparks discussions and partnerships that continue to enrich the field of endangered language research, ultimately contributing to more robust support for language communities worldwide.